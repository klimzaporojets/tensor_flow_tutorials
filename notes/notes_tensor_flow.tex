 \documentclass[11pt,a4paper]{article}
\usepackage[utf8]{inputenc}
\usepackage{amsmath}
\usepackage{amsfonts}
\usepackage{amssymb}
\usepackage{courier}
\usepackage{listings}
\usepackage{color}

\definecolor{mygreen}{rgb}{0,0.6,0}
\definecolor{mygray}{rgb}{0.5,0.5,0.5}
\definecolor{mymauve}{rgb}{0.58,0,0.82}

\lstset{ %
  backgroundcolor=\color{white},   % choose the background color; you must add \usepackage{color} or \usepackage{xcolor}; should come as last argument
  basicstyle=\footnotesize,        % the size of the fonts that are used for the code
  breakatwhitespace=false,         % sets if automatic breaks should only happen at whitespace
  breaklines=true,                 % sets automatic line breaking
  captionpos=b,                    % sets the caption-position to bottom
  commentstyle=\color{mygreen},    % comment style
  deletekeywords={...},            % if you want to delete keywords from the given language
  escapeinside={\%*}{*)},          % if you want to add LaTeX within your code
  extendedchars=true,              % lets you use non-ASCII characters; for 8-bits encodings only, does not work with UTF-8
  frame=single,	                   % adds a frame around the code
  keepspaces=true,                 % keeps spaces in text, useful for keeping indentation of code (possibly needs columns=flexible)
  keywordstyle=\color{blue},       % keyword style
  language=python,                 % the language of the code
  morekeywords={*,...},           % if you want to add more keywords to the set
  numbers=left,                    % where to put the line-numbers; possible values are (none, left, right)
  numbersep=5pt,                   % how far the line-numbers are from the code
  numberstyle=\tiny\color{mygray}, % the style that is used for the line-numbers
  rulecolor=\color{black},         % if not set, the frame-color may be changed on line-breaks within not-black text (e.g. comments (green here))
  showspaces=false,                % show spaces everywhere adding particular underscores; it overrides 'showstringspaces'
  showstringspaces=false,          % underline spaces within strings only
  showtabs=false,                  % show tabs within strings adding particular underscores
  stepnumber=2,                    % the step between two line-numbers. If it's 1, each line will be numbered
  stringstyle=\color{mymauve},     % string literal style
  tabsize=2,	                   % sets default tabsize to 2 spaces
  title=\lstname                   % show the filename of files included with \lstinputlisting; also try caption instead of title
}
\author{Klim Zaporojets}
\title{Notes on udacity deep learning course}
\setlength\parindent{0pt}
\begin{document}
\section{ML For Beginners}
softmax:
\begin{enumerate}
	\item If you want to assign probabilities to an object being one of several different things, softmax is the thing to do, because softmax gives us a list of values between 0 and 1 that add up to 1. Even later on, when we train more sophisticated models, the final step will be a layer of softmax.
\end{enumerate}

$$\text{softmax}(x)_i = \dfrac{\exp(x_i)}{\sum_j{\exp(x_j)}}$$

\section{Variables: Creation, Initialization, Saving, and Loading}
When you train a model, you use variables to hold and update parameters. Variables are in-memory buffers containing tensors. They must be explicitly initialized and can be saved to disk during and after training. You can later restore saved values to exercise or analyze the model.
\texttt{tf.Variable} is used to create variable. Initialization function should be provided (ex: \texttt{tf.random\_normal}) as well as the shape of the resulting tensor. 
\begin{lstlisting}. Example: 
# Create two variables.
weights = tf.Variable(tf.random_normal([784, 200], stddev=0.35),
                      name="weights")
biases = tf.Variable(tf.zeros([200]), name``biases")
\end{lstlisting}
Variables can also be placed on a device: 
\begin{lstlisting}. Example: 
# Pin a variable to GPU.
with tf.device(``/gpu:0"):
  v = tf.Variable(...)
\end{lstlisting} 
In order to initialize variables \texttt{tf.initialize\_all\_variables()} has to be used to add op to run variables initializers on a particular session. \\
Use \texttt{initialized\_value()} property to initialize a variable from the initial value of another variable. \\
To save and restore variable values, \texttt{tf.train.Saver} object has to be used. This creates \textit{checkpoint files}. 

\section{TensorBoard}
The computations you'll use TensorFlow for - like training a massive deep neural network - can be complex and confusing. To make it easier to understand, debug, and optimize TensorFlow programs, we've included a suite of visualization tools called TensorBoard. You can use TensorBoard to visualize your TensorFlow graph, plot quantitative metrics about the execution of your graph, and show additional data like images that pass through it. \\
In order to produce logs, the data for each of the summary stats has to be summarized via \texttt{SummaryWriter} and written to the disk to a particular location that will have to be passed to the parameter \texttt{logdir} when loading tensorboard. 

To see different tensorboard  visualizations execute: \\ 
\texttt{tensorboard --logdir=path/to/log-directory}

\subsection{TensorBoard Graph visualization}
In case of dealing with lots of nodes, scoping can be an interesting option: 
\begin{lstlisting}. Example: 
import tensorflow as tf

with tf.name_scope('hidden') as scope:
  a = tf.constant(5, name='alpha')
  W = tf.Variable(tf.random_uniform([1, 2], -1.0, 1.0), name='weights')
  b = tf.Variable(tf.zeros([1]), name='biases')
  
  \end{lstlisting} 
\section{Tensorflow mechanics}
Possible to save tensorflow checkpoints using \texttt{tf.train.Saver} class.\\
\textbf{\texttt{tf.placeholder}}: inserts a placeholder for a tensor that will be always fed. 
\begin{lstlisting}. Example: 
images_placeholder = tf.placeholder(tf.float32, 
	shape=(batch_size, mnist.IMAGE_PIXELS))
labels_placeholder = tf.placeholder(tf.int32, shape=(batch_size))
\end{lstlisting}
Grouping nodes by name scopes is critical to making a legible graph. If you're building a model, name scopes give you control over the resulting visualization. \textbf{The better your name scopes, the better your visualization}.

\section{Reading data}

\end{document}